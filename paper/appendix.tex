\section{The background torque density in a three-dimensional disc with
  a fixed temperature profile}\label{tbg_deriv}
We give a brief derivation of the angular momentum exchange between
linear perturbations and the background disc. We consider a
three-dimensional disc in which the equilibrium pressure and density
are related by 
\begin{align}\label{iso_cond}
  p = c_s^2(R,z)F(\rho),
\end{align} 
where $F(\rho)$ is an arbitrary function of $\rho$ with dimensions of
mass per unit volume, and $c_s$ is a prescribed function of
position with dimensions of velocity squared. The equilibrium disc
satisfies 
\begin{align}
  R\Omega^2(R,z) &= \frac{1}{\rho}\frac{\p p}{d R} +
  \frac{\p\Phi_\mathrm{tot}}{\p R}, \\
  0 &= \frac{1}{\rho}\frac{\p p}{\p z} + \frac{\p\Phi_\mathrm{tot}}{\p
    z}. 
\end{align}
Note that, in general, the equilibrium rotation $\Omega$ depends on
$R$ and $z$. 

We begin with the linearised equation of motion in terms of the
Lagragian displacement $\bm{\xi}$ as given by \cite{lin93b} but with an
additional potential perturbation, 
\begin{align}\label{lagragian_pert}
  &\frac{D^2\bm{\xi}}{Dt^2} +
  2\Omega\hat{\bm{z}}\times\frac{D\bm{\xi}}{Dt}  \notag \\ &= -
  \frac{\nabla \delta p }{\rho} + \frac{\delta\rho}{\rho^2}\nabla\rho  
  -\nabla\delta\Phi_d - R
  \hat{\bm{R}}\left(\bm{\xi}\cdot\nabla\Omega^2\right) \notag \\
  & = -\nabla\left(\frac{\delta p}{\rho} + \delta \Phi_d\right) -
  \frac{\delta p}{\rho}\frac{\nabla\rho}{\rho} +
  \frac{\delta\rho}{\rho}\frac{\nabla p}{\rho} - - R
  \hat{\bm{R}}\left(\bm{\xi}\cdot\nabla\Omega^2\right),
\end{align}
where $D/Dt \equiv \p_t + \ii m \Omega$ for perturbations with
azimuthal dependence in the form $\exp\left(\ii m \phi\right)$, and  
the $\delta$ quantities denote Eulerian perturbations. 

As explained in \cite{lin11b}, a conservation law for the angular
momentum of the perturbation may be obtained by taking the dot product
between Eq. \ref{lagragian_pert} and $(-m/2)\rho\bm{\xi}^*$, then
taking the imaginary part afterwards. The left hand side bcomes the
angluar momentum density. The first term on the right hand side (RHS)
becomes 
\begin{align}\label{angflux1}
  &-\frac{m}{2}\imag\left[-\rho\bm{\xi}^*\cdot\nabla\left(\frac{\delta p}{\rho} + \delta
    \Phi_d\right)\right] \notag\\ 
&= \frac{m}{2}\imag\left\{\nabla\cdot\left[\rho\bm{\xi}^*\left(\frac{\delta p}{\rho} + \delta
    \Phi_d \right) + \frac{1}{4\pi
    G}\delta\Phi_d\nabla\delta\Phi_d^*\right]\right\} \notag\\
&+ \frac{m}{2}\imag\left(\delta\rho^*\frac{\delta p}{\rho}\right),
\end{align}
where $\delta\rho = - \nabla\cdot\left(\rho\bm{\xi}\right)$ and
$\nabla^2\delta\Phi_d = 4\pi G \delta \rho$ have been used. The first
term on the RHS of Eq. \ref{angflux1} is (minus) the 
angular momentum flux. The second term on RHS of Eq. \ref{angflux1}, 
together with the remaining terms on the RHS of 
Eq. \ref{lagragian_pert} constitutes the background torque. That is, 
\begin{align}\label{tbg_3d}
  T_\mathrm{BG} = \frac{m}{2}\imag\left[
    \frac{\delta p}{\rho} \Delta\rho^* -
%\left(\delta\rho^* + \bm{\xi}^*\cdot\nabla\rho\right) - 
    \frac{\delta\rho}{\rho}\bm{\xi}^*\cdot\nabla p  
    + \rho \xi_R^*\xi_z\frac{\p\left(R\Omega^2\right)}{\p z}\right],
\end{align}
where $\Delta\rho = \delta\rho + \bm{\xi}\cdot\nabla\rho$ is the
Lagragian density perturbation. 

So far we have not invoked an energy equation. For 
adiabatic perturbations $T_\mathrm{BG}$ is zero
\citep{lin93b}. However, if we impose the equilibrium relation
Eq. \ref{iso_cond} to hold in the perturbed state, then
\begin{align}
  \delta p = c_s^2(R,z) F^\prime \delta\rho,
\end{align}
where $F^\prime = dF/d\rho$. Inserting this into Eq. \ref{tbg_3d}, we
obtain
\begin{align}\label{tbg_3d_2}
  T_\mathrm{BG} = -\frac{m}{2}\frac{p}{\rho
    c_s^2}\imag\left[\delta\rho\bm{\xi}^*\cdot\nabla c_s^2 +
    \xi_R^*\xi_z \left(\frac{\p\rho}{\p z}\frac{\p c_s^2}{\p R} -
      \frac{\p\rho}{\p R}\frac{\p c_s^2}{\p z}\right)\right],
\end{align}
where the equilibrium equations were used. 
At this point setting $\xi_z=0$ gives $T_\mathrm{BG}$ for
perturbations with no vertical motion, 
\begin{align}
  T_\mathrm{BG,2D} = -\frac{m}{2}\frac{p}{\rho
    c_s^2}\imag\left(\delta\rho \xi_R^*  \p_R c_s^2 \right),
\end{align}
and is equivalent to the 2D
expression, Eq. \ref{baroclinic_torque}, with $\delta\rho $ replaced
by $\delta \Sigma$ and $p=c_s^2\rho$. % The 2D expression is convenient
% for razor-thin discs or perturbations without vertical motion because
% one can measure $\delta\rho$ (or $\delta \Sigma)$ directly.  


In fact, we can simplify Eq. \ref{tbg_3d_2} in the general case by
using $\delta\rho = - \rho\nabla\cdot\bm{\xi} -
\bm{\xi}\cdot\nabla\rho$, giving
\begin{align}\label{tbg_general}
  T_\mathrm{BG} = \frac{m}{2}\frac{p}{\rho
    c_s^2}\imag\left[\rho\left(\nabla\cdot\bm{\xi}\right)\bm{\xi}^*\cdot\nabla
  c_s^2\right].  
\end{align} 
For a barotropic fluid $p=p(\rho)$, the function $c_s^2$ can be
taken as constant (Eq. \ref{iso_cond}) for which $T_\mathrm{BG}$
vanishes. When there is a forced temperature gradient,
Eq. \ref{tbg_general} indicates there is angular momentum exchange
between compressible perturbations ($\nabla\cdot\bm{\xi}\neq0$) and
the background disc if there is motion along the temperature
gradient ($\bm{\xi}\cdot\nabla c_s^2 \neq 0$).  

% total effect needs to be integrated

% For the perturbations without vertical motion or those in razor-thin
% discs, the expression Eq. \ref{}

%eq baroclinic torque is more practical in 2d case because \delta
%sigma can be measured direction from simulations (xi needs to be
%calculated from velocity field, knowing the eigenfrequency)