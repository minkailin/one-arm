%to discuss - fixed central star
%application to ppd
%do smooth disc results carry over to structured disks?

\section{Discussion}\label{discussions} 
{\bf
  We discuss below several issues related to self-gravitating
  discs in the context of our numerical simulations. However, it is
  important to keep in mind that, strictly speaking, 
  the growth of the one-armed spiral in our simulations is
  \emph{not} a gravitational instability in the sense that
  destabilisation is through the background torque associated with a
  forced temperature gradient, and not by self-gravitational torques.     
}

{\bf
  \subsection{Motion of the central star} 
  In our models we have purposefully neglected the indirect potential
  associated with a non-inertial reference frame to avoid
  complications arising from the motion of the central star.  
  It has been established that such motion, in response to an $m=1$
  disturbance in the disc, can render the disturbance
  unstable in a barotropic disc (so the background torque is
  ineffective). 

  However, for the disc masses in our models, $M_d\lesssim 
  0.1 M_*$, this effect is not expected to be significant
  \citep{adams89,shu90,michael10}. Indeed, simulations including the
  indirect potential, carried out in the early stages of this project,
  produced similar results.    
  
  \subsection{Role of Lindblad and co-rotation torques}
  One effect of self-gravity is to allow the one-armed spiral,
  confined to $R\sim R_0$, to act as an external potential for the
  exterior disc in $R\gtrsim R_0$. This is 
  analogous to disc-satellite interaction 
  \citep{goldreich79}, where the embedded satellite exerts a torque on
  the disc at Lindblad and co-rotation resonances. 

  In Appendix \ref{disc-planet} we estimate the magnitude of this effect
  using basic results from disc-planet theory \citep[see, e.g.][and
  references therein]{papaloizou07}. There, we find that the angular
  momentum exchange between the one-armed spiral and the exterior disc
  is insignificant compared to the background torque.  
  
  We confirmed this with additional FARGO simulations that exclude the
  co-rotation and outer Lindblad resonances (OLR) by reducing radial domain
  size, setting $R_\mathrm{max} = 3 R_0$. This model still developed
  the one-armed spiral. 
}


{\bf
  \subsection{Applicability to circumstellar discs}
  
  \subsubsection{Thermodynamic requirements}  
  % talk about additional simulations at high Q
  %issues with fragmentation 
 % requirement for fast cooling  
  % fast cooling fragments 
  % stabilised by irradiation
  % background torque and fragmentation may not be mutually exclusive
  % (speculations) 
  % these studies do no account for possible global disk structure 
  %Clarke (2009), Rafikov (2005), and Rice & Armitage (2009)
  %no fragmentation in our case because disc is maintained at slightly
  %higher values of Q. no fragmentation of one arm spiral because
  %co-rotation radius isn't on it 
  %if irradiation 
  %one-armed low frequency spiral probably will not fragment because
  %co-rotation is far away 

  A locally isothermal equation of state represents the limit
  of infinitely short cooling (and heating) timescales, so the 
  disc temperature instantly returns to its initial value when 
  perturbed. The background torque is generally non-zero if the 
  resulting temperature profile has a non-zero radial gradient.  
  
  A short cooling timescale $t_c$ can occur in the outer
  parts of protoplanetary discs 
  \citep{rafikov05,clarke09,rice09,cossins10b}.  %rafikov07 (convection)
  However, if a disc with $Q\simeq 1$ is cooled (towards zero
  temperature) on a timescale $t_c\lesssim\Omega_k^{-1}$, it will
  fragment following gravitational instability
  \citep{gammie01,rice05,paardekooper12}.  
  
  Fragmentation can be avoided if the disc is heated to maintain 
  $Q>Q_c$, the threshold for fragmentation \citep[$Q_c\simeq
  1.4$ for isothermal discs,][]{mayer04}. This may be 
  possible in the outer parts of protoplanetary discs due to
  stellar irradiation \citep{rafikov09,zhu12}. Sufficiently strong
  external irradiation is expected to suppress the linear gravitational
  instabilities altogether \citep{rice11}.  

  %chiang and goldreich 
  The background torque may thus exist in the outer
  parts of protoplanetary discs that are strongly irradiated, such
  that the disc temperature is set externally with a non-zero radial
  gradient. Of course, if external irradiation sets a strictly
  isothermal outer disc \citep[e.g.][]{boley09}, then the background
  torque vanishes.      

  % We note, however, that there is no obvious reason why the background
  % torque and gravitational instability/fragmentation cannot coexist.  

  % It is important to note, however,
  % that these studies do not consider the possible effect of global
  % structures present in our numerical simulations.  
  
  % When $Q\simeq 1$ and the cooling timescale
  % $t_c\gtrsim\Omega_k^{-1}$, the disc reaches a
  % quasi-steady in which cooling is balanced by heating due to
  % gravito-turbulence \citep{lodato04}. This would, in general, produce
  % a global radial temperature gradient.   
  
  % There
  % is no real way to drive a long-lived quasi-steady, non-fragmenting,
  % self-gravitating state because this is only possible if the system
  % can remain quasi-stable though balancing heating and cooling (see,
  % for example, Lodato & Rice 2004).   
  
  % where does the structure come from? (mass built up, planet gap, infall) 
 
  \subsubsection{Radial disc structure}
  In our simulations the 
  $m=1$ spiral is confined between two $Q$-barriers, where real solutions to the local
  dispersion relation is possible (Eq. \ref{wavenumber}). The
  existence of such a cavity is a result of the initial surface
  density bump (Eq. \ref{sig_bump}).  
  Thus, in our disc models the main role of disc structure 
  and self-gravity is to allow a local $m=1$ mode to be set up, which is then 
  destabilised by the background torque. 
  
  In order to confine an $m=1$ mode between two $Q$-barriers, we
  should have $Q^2(1-\nu^2)=1$ at two radii. Assuming 
  Keplerian rotation and a slow pattern speed $\Omega_p\ll\Omega$,
  this amounts to 
  \begin{align}\label{qb_cond}
    \left(\frac{R_{Qb}}{R_c}\right)^{-3/2} = 2Q^2(R_{Qb}). 
  \end{align}
  Then two $Q$-barriers may exist when the $Q$ profile
  rises more rapidly (decays more slowly) than $R^{-3/2}$
  for decreasing (increasing) $R$. 
  
  A surface density bump can develop in 
  `dead zones' of protoplanetary 
  discs, where there is reduced mass accretion because the magneto-rotational 
  instability is ineffective for angular momentum transport 
  \citep{gammie96,turner08,landry13}. The dead zone becomes 
  self-gravitating with sufficient mass built-up 
  \citep{armitage01,martin12,martin12b,zhu09,zhu10,zhu10b,bae13}.  
  
  However, conditions in a dead zone may not be suitable for 
  sustaining a background torque because it may not cool/heat fast enough
  to maintain a fixed temperature profile. Recently, 
  \cite{bae14} presented sophisticated numerical models of
  dead zones, including a range of heating and 
  cooling processes, which show that dead zones developed large-scale
  (genuine) gravitational instabilities with multiple spiral
  arms. Although this does not prove absence of the background torque,
  it is probably insignificant compared to gravitational torques. 

%  There is some attempt to do so when discussing dead zones, but I
%  doubt that the conditions there are actually suitable as such a
%  region would either be stable, or cool very slowly and be
%  susceptible to the growth of long-lived, high-m-mode spiral
%  structure. 

  Another possibility is a gap opened by an embedded planet. In that case $Q$
  rises rapidly towards the gap centre since it is a region of low
  surface density. This can satisfy Eq. \ref{qb_cond}. Then the inner
  edge of our bump function mimics the outer gap edge. The outer gap
  edge is then a potential site for the growth of a low-frequency
  one-armed spiral through the background torque. However, the
  locally isothermal requirement limits this process to the
  outer disc.   
}


% The simulations themselves consider an isothermal or locally
% isothermal equation of state only.  Gammie (2001) and other work
% since have shown that the evolution of self-gravitating discs
% depends both on the stability criterion and on the balance between
% heating and cooling.  If the cooling rate is fast, it is likely to
% fragment, otherwise it will settle into a quasi-steady,
% self-gravitating state.     

% When considering isothermal simulations, the likely evolution is one
% of two things.  It's either initially so unstable that the disc
% quickly fragments, or it's initially stable enough that any initial
% perturbations simply drives the system towards being even more
% stable and the instability quickly weakens through rapid mass
% transfer (see the work by Laughlin & Bodenheimer 1994/1996).  There
% is no real way to drive a long-lived quasi-steady, non-fragmenting,
% self-gravitating state because this is only possible if the system
% can remain quasi-stable though balancing heating and cooling (see,
% for example, Lodato & Rice 2004).   

% In fact, in the conclusions it is suggested that the conditions used
% in the simulation would require a cooling < 0.1 \Omega^{-1} which,
% based on the work of Gammie 2001, would quickly lead to
% fragmentation.  The reason it doesn't in the work considered here is
% that the isothermal assumption assumes that there is some
% equivalently fast heating mechanism but, in the work here, this is
% not provided by the instability itself.  If it were, the simulations
% would become very quickly unstable.  Also, if one looks at the work
% of Clarke (2009), Rafikov (2005), and Rice & Armitage (2009) the
% only region of a realistic disc where such rapid cooling is possible
% is in the outer disc, but such regions are either very
% gravitationally unstable and fragment, or quickly stablised by an
% external heating source.  I guess it is possible that there could be
% a scenario where this exactly produces the conditions assumed here,
% but that would seem rather fine-tuned and it would be more likely
% that an external heating source would simply stabilise the disc
% (unless the disc were extremely massive and could remain unstable
% even at reasonably high disc temperatures).  


























%Furthermore, this
%non-local interaction may be important for the long-term evolution of
%the outer disc, since it gains angular momentum from the

%also transport ang mom outwards


% Having confirmed the $m=1$ spiral instability through numerical
% simulations, we now use results from the simulations combined with
% linear theory to construct an explanation for their growth. For
% simplicity, we return to 2D discs. 

% The dominant disturbance in our simulations is an one-armed, $m=1$
% trailing spiral, and its growth is associated with the imposed temperature
% gradient. As discussed in \S\ref{global_cons}, in a locally isothermal disc this imposed
% temperature gradient generally results in a torque being exerted on
% linear perturbations. An instability is therefore possible if the sign
% of this torque is the same as that of the angular momentum associated
% with the linear perturbation. 

%associated angular mometum density is the same as that
%of this
% According to previous studies \citep{adams89}, our disc models
% should not support $m=1$ linear spiral instabilities. This is because
% our discs are not sufficiently massive ($M_d\lesssim 0.1M_*$) and, more
% importantly, we have surpressed the motion of the central star. The 
% latter can induce $m=1$ instabilities in massive discs  \citep{shu90}.   


%the latter assumes  k=pi\Sigma G /c_s^2 is always the most favoured 

% The temperature gradient affects the growth rate through the
% explicit depdence of the background torque on $q$. The growth 
% rate is also dependent on the underlying disturbance being
% considered (i.e. through $k$ and $\bm{\xi}$ in
% Eq. \ref{theoretical_rate0}). % Of course, the temperature gradient
% % plays no role on local modes. However, other disc parameters such as
% % $c_s^2$ may affect the mode that develops. 
% The dependence on $h$ in 
% Eq. \ref{theoretical_rate} assumes $k=\pi\Sigma G/c_s^2$ is always the
% most favourable wavenumber for low-frequency modes.  
 
%By 
%definition, the global temperature gradient does not affect modes in
%local theory. 
%however, changing $h$ may also modify the mode properties (since
%temperature appears in local theory) in addition
%to the magnitude of the temperature gradient. Thus, we expect the depedence of
%the background torque on $q$ to be better described by
%Eq. \{theoretical_rate} 

%by definition, q does not affect local neutral modes
%Of course, we assumed a neutral mode to obtain expressions
%for $\jlin$ and $T_\mathrm{BG}$ in the first place, 
%Recall that in local theory non-axisymmetric modes are stable for real
%$k$  
%Eq. \ref{theoretical_rate} should be interpreted
%as an order-of-magnitude estimate of the perturbative effect of a
%temperature gradient on local, low-frequency $m=1$ modes.  

%neutral modes are not actually neutral)
%gamma << omega_p << omega 
%so
%Eq. \ref{theoretical_rate} is only an order-of-magnitude
%constraint. Nevertheless, the expected growth rate is in reasonable
%agreement with numerical results. 

%also total torque is integral -> may get cancelations 
