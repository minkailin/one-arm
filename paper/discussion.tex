\section{Discussion}
Having confirmed the $m=1$ spiral instability through numerical
simulations, we now use results from the simulations combined with
linear theory to construct an explanation for their growth. For
simplicitly, we return to 2D discs. 

\subsection{Angular momentum density of  low-frequency $m=1$ modes}
From Eq. \ref{lin_ang_mom_cons} and assuming a time-dependence of the
form $\exp{(-\ii \sigma t)}$ with $\real{\sigma} = \omega$,  
the angular momentum density associated with linear waves is
\begin{align}
  \jlin = \frac{m\Sigma}{2}\left[\left(\omega -
      m\Omega\right)|\bm{\xi}|^2 + \ii\Omega\left(\xi_R\xi_\phi^* -
      \xi_R\xi_\phi^*\right)\right].  
\end{align}
For a low-frequency mode, $|\omega|\ll m\Omega$. Then neglecting the
term $\omega|\bm{\xi}|^2$ and setting $m=1$,
\begin{align}
  \jlin &\simeq \frac{\Sigma\Omega}{2}\left[-|\bm{\xi}|^2 + \ii\left(\xi_R\xi_\phi^* -
      \xi_R\xi_\phi^*\right)\right]\notag\\
  & = -\frac{\Sigma\Omega}{2}|\xi_R + \ii\xi_\phi|^2\notag\\
  & \leq 0.
\end{align}
Thus, low-frequency $m=1$ modes are generally associated with negative
angular momentum. If the mode loses (positive) angular momentum
to the background, then we can expect instability. We demonstrate
this below. 

\subsection{Unstable interaction between low-frequency $m=1$ spirals
  and the background disc due to an imposed temperature gradient}
The $m=1$ spiral that emerges in our simulations have low frequency
and grows slowly, and its spatial form is consistent local theory
(\S\ref{fargo_m1}). To demonstrate possible instability, we
first construct a neutral low-frequency mode using local theory, then
show that the torque density acting on this mode due to the background
temperature gradient is negative, which enforces the mode. % Note that
% we will not be assuming a self-gravitating disc until the end of this
% section. 

We begin by recalling the expression for the torque density due to a
locally isothermal equation of state,
\begin{align}
  T_\mathrm{BG} &= -\frac{m}{2}\frac{dc_s^2}{dR}\imag\left[\delta\Sigma_m \left(\frac{\delta
        v_{Rm}}{-\ii\sbar}\right)^*\right] \notag\\
  & = \frac{m}{2}\frac{dc_s^2}{dR}\imag\left[\ii \delta\Sigma_m \left(\frac{\delta
        v_{Rm}}{\sbar}\right)^*\right]
\end{align}
where we replaced the radial Lagrangian displacement by the Eulerian
radial velocity perturbation in Eq. \ref{baroclinic_torque}. 
The linearised momentum equations give
\begin{align}
  D\delta v_{Rm} =& 
  \ii\sbar\left[c_s^2\frac{d}{dR}\left(\frac{\delta\Sigma_m}{\Sigma}\right)
    + \frac{d}{dR}\delta\Phi_m\right] \notag\\ 
  &- \frac{2\ii
    m\Omega}{R}\left(c_s^2\frac{\delta\Sigma_m}{\Sigma} +
    \delta\Phi_m\right),
\end{align}
where $D\equiv \kappa^2 - \sbar^2$. 

We now invoke local theory by setting $d/dR \to \ii k$ where $k$ is
real, and using the local solution to the Poisson equation
\begin{align}
  \delta\Phi_m = -\frac{2\pi G}{|k|}\delta\Sigma_m 
\end{align}
\citep{shu91}. Then the radial velocity perturbation becomes
\begin{align}
  \delta v_{Rm} =
  \frac{1}{D}\frac{\delta\Sigma_m}{\Sigma}\left(\frac{2\pi G
      \Sigma}{|k|} - c_s^2\right)\left(k\sbar + \frac{2\ii
      m\Omega}{R}\right),
\end{align}
so that 
\begin{align}
  \ii\delta\Sigma_m \left(\frac{\delta v_{Rm}}{\sbar}\right)^* =
  \frac{1}{D^*}\frac{|\delta\Sigma_m|^2}{\Sigma}\left(\frac{2\pi G
      \Sigma}{|k|} - c_s^2\right)\left(\ii k  + \frac{2
      m\Omega}{R\sbar^*}\right). 
\end{align}
For a neutral mode, $\sbar$, and therfore $D$ is real. Then, after
taking the imaginary part of the above expression, the torque
density is 
\begin{align}\label{tbg_explicit}
  T_\mathrm{BG} &=
  \frac{m}{2}\frac{dc_s^2}{dR}\left[\frac{k}{D}\frac{|\delta\Sigma_m|^2}{\Sigma}\left(\frac{2\pi G
      \Sigma}{|k|} - c_s^2\right)\right].
% &\simeq \frac{m}{2}\frac{dc_s^2}{dR}\left[\frac{k}{\left(\kappa^2 -
%         m^2\Omega^2\right)}\frac{|\delta\Sigma_m|^2}{\Sigma}\left(\frac{2\pi
%       G \Sigma}{|k|} - c_s^2\right)\right]. 
\end{align}
%where the second line follows from the low-frequency approximation
%$|\sigma|\ll m\Omega$. 
% To determine the sign of $T_\mathrm{BG}$ for low frequency one-arm spirals,
% we examine each factor separately below for $m=1$:
% \begin{itemize}
% \item We observe trailing spirals in our simulations, so $k>0$. 
% \item The factor $\kappa^2 - \Omega^2>0$ for our self-gravitating disc
%   models in the region where the spiral appears. 
% \item We measure $|k|\sim \pi G\Sigma/c_s^2$, so the last factor in
%   the square brackets is positive. Note that this factor is positive
%   provided $|k| < 2/HQ$ or wavelengths $\lambda > \pi Q H$.  
% \end{itemize}
We now use the local dispersion relation,
Eq. \ref{dispersion}, to make the replacement $D = 2\pi G \Sigma |k| -
k^2c_s^2$ in the above expression. This gives
\begin{align}\label{tbg_simple}
  T_\mathrm{BG} =
  \frac{m}{2}\frac{dc_s^2}{dR}\frac{1}{k}\frac{|\delta\Sigma_m|^2}{\Sigma}.  
\end{align} 
This torque density is negative for trailing waves in discs with
temperature decreasing outwards ($k>0$ and $dc_s^2/dR<0$,
respectively). Note that this conclusion does not rely on the
low-frequency approximation and holds for any $m$.   

However, if the linear disturbance under consideration \emph{is} a
low-frequency $m=1$ 
trailing spiral wave, then it has negative angular
momentum. If $dc_s^2/dR<0$, as is typical for circumstellar or
protoplanetary discs, then $T_\mathrm{BG}<0$, and the background disc
applies a negative torque on the disturbance, which further decreases
its angular momentum. This suggests the spiral amplitude will grow.  


Using $\jlin$ and $T_\mathrm{BG}$, we can obtain a rough estimate of
the growth rate $\gamma$ of linear perturbations as  
\begin{align}
  2\gamma \sim \frac{T_\mathrm{BG}}{\jlin},
\end{align}
where the factor of two accounts for the fact that the angular momentum
density is quadradtic in the linear perturbations. Inserting the above
expressions for $\jlin$ and $T_\mathrm{BG}$ for $m=1$ gives
\begin{align}
  2\gamma \sim
  -\frac{dc_s^2}{dR}\frac{1}{k\Sigma^2\Omega}\frac{|\delta\Sigma_1|^2}{|\xi_R
    + \ii\xi_\phi|^2}.
\end{align}
Now, using $\delta\Sigma_m/\Sigma = -\ii k \xi_R - \ii m \xi_\phi/R$
in the local approximation, we have
\begin{align}
  2\gamma \sim -\frac{dc_s^2}{dR}\frac{k}{\Omega}\frac{|\xi_R +
    \xi_\phi/kR|^2}{|\xi_R + \ii \xi_\phi|^2}
  \lesssim -\frac{dc_s^2}{dR}\frac{k}{\Omega}.
\end{align}
where we used $dc_s^2/dR <0$, $k>0$ and $|kR|\gg 1$ to obtain the
inequality. For $c_s^2 = c_{s0}^2 (R/R_0)^{-q}$ adopted in our disc
models,
\begin{align}\label{theoretical_rate}
  2\gamma \lesssim q\frac{c_s^2}{R}\frac{k}{\Omega}. 
\end{align}
Now we consider perturbations in a self-gravitating disc with characteristic
wavenumber $k \sim \pi G \Sigma/c_s^2 \sim \Omega/Q c_s$ (as seen in
our simulations). Then
\begin{align}
  \gamma \lesssim \frac{qh}{2Q}\Omega,
\end{align}
where we used $c_s/R\sim h\Omega$. For our fiducial disc models with
$q=1$, $Q=O(1)$, $h=0.05$, this gives $\gamma/\Omega = O(10^{-2})$,
consistent with simulation results. 

Of course, we have been assuming a neutral mode to obtain expressions
for $\jlin$ and $T_\mathrm{BG}$ in the first place, so the above can only be
considered as an order-of-magnitude estimate. Nevertheless, the
expected growth rate is in reasonable agreement with numerical
results. 

\subsection{Role of self-gravity and disc structure} 
We note that the background torque density $T_\mathrm{BG}$ does not depend on  
self-gravity, and we only considered a self-gravitating perturbation to
obtain an appropriate wavenumber $k$ needed to evaluate
Eq. \ref{theoretical_rate}. Thus, self-gravity is not directly
responsible for instability. However, 
our numerical simulations show that the $m=1$ spiral is
confined between two $Q$-barriers, where real solutions to the local
dispersion relation is possible (Eq. \ref{wavenumber}), as a result of the adopted disc
structure. Thus, the main role of disc structure and self-gravity is to
allow a neutral mode to be set up and confined, which is then
destabilised through the background torque density. 

%talk about additional simulations at high Q
%spiral mode as external potential to wider disc (lp11)
%interaction across co-rotation

\subsubsection{The confined spiral as an external potential}
One possible effect of self-gravity is to allow the one-arm spiral 
to act as an external potential for the wider disc. This effect is
analogous to disc-satellite interaction
\citep{goldreich79}, and may lead to instability 
if the angular momentum associated with the spiral has the opposite 
sign to that of the density waves it induces in the exterior disc
\citep{lin11b}. 

Considering the spiral as a external potential (c.f. an embedded
satellite), we expect it to  exert a torque on the disc at Lindblad
and co-rotation resonances. 

%speculations - long term balance. forced temp gradient drives
%instability (disc gains ang mom, spiral loses ang mom). but if spiral
%shocks and dissiple, it dumps negative angular momentum (disc gains
%negative ang mom). balance between gain from forced temp gradient and
%loss due to dissiplatoin