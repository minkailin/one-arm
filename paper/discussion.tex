%to discuss - fixed central star
%application to ppd
%do smooth disc results carry over to structured disks?

\section{Discussion}\label{discussions} 



% Having confirmed the $m=1$ spiral instability through numerical
% simulations, we now use results from the simulations combined with
% linear theory to construct an explanation for their growth. For
% simplicity, we return to 2D discs. 

% The dominant disturbance in our simulations is an one-armed, $m=1$
% trailing spiral, and its growth is associated with the imposed temperature
% gradient. As discussed in \S\ref{global_cons}, in a locally isothermal disc this imposed
% temperature gradient generally results in a torque being exerted on
% linear perturbations. An instability is therefore possible if the sign
% of this torque is the same as that of the angular momentum associated
% with the linear perturbation. 

%associated angular mometum density is the same as that
%of this
% According to previous studies \citep{adams89}, our disc models
% should not support $m=1$ linear spiral instabilities. This is because
% our discs are not sufficiently massive ($M_d\lesssim 0.1M_*$) and, more
% importantly, we have surpressed the motion of the central star. The 
% latter can induce $m=1$ instabilities in massive discs  \citep{shu90}.   


%the latter assumes  k=pi\Sigma G /c_s^2 is always the most favoured 

% The temperature gradient affects the growth rate through the
% explicit depdence of the background torque on $q$. The growth 
% rate is also dependent on the underlying disturbance being
% considered (i.e. through $k$ and $\bm{\xi}$ in
% Eq. \ref{theoretical_rate0}). % Of course, the temperature gradient
% % plays no role on local modes. However, other disc parameters such as
% % $c_s^2$ may affect the mode that develops. 
% The dependence on $h$ in 
% Eq. \ref{theoretical_rate} assumes $k=\pi\Sigma G/c_s^2$ is always the
% most favourable wavenumber for low-frequency modes.  
 
%By 
%definition, the global temperature gradient does not affect modes in
%local theory. 
%however, changing $h$ may also modify the mode properties (since
%temperature appears in local theory) in addition
%to the magnitude of the temperature gradient. Thus, we expect the depedence of
%the background torque on $q$ to be better described by
%Eq. \{theoretical_rate} 

%by definition, q does not affect local neutral modes
%Of course, we assumed a neutral mode to obtain expressions
%for $\jlin$ and $T_\mathrm{BG}$ in the first place, 
%Recall that in local theory non-axisymmetric modes are stable for real
%$k$  
%Eq. \ref{theoretical_rate} should be interpreted
%as an order-of-magnitude estimate of the perturbative effect of a
%temperature gradient on local, low-frequency $m=1$ modes.  

%neutral modes are not actually neutral)
%gamma << omega_p << omega 
%so
%Eq. \ref{theoretical_rate} is only an order-of-magnitude
%constraint. Nevertheless, the expected growth rate is in reasonable
%agreement with numerical results. 

%also total torque is integral -> may get cancelations 




\subsection{The role of self-gravity and disc structure} 
It is important to note that in the above discussion, we only considered a self-gravitating
perturbation to obtain an appropriate wavenumber $k$ needed to
evaluate Eq. \ref{theoretical_rate}. 
The background torque density $T_\mathrm{BG}$ does not depend on  
self-gravity explicitly, so self-gravity is not directly
responsible for instability. 

However, 
our numerical simulations show that the $m=1$ spiral is
confined between two $Q$-barriers, where real solutions to the local
dispersion relation is possible (Eq. \ref{wavenumber}), as a result of the adopted disc
structure. Thus, in our disc models the main role of disc structure and self-gravity is to 
allow a neutral mode to be set up and confined, which is then
destabilised through the background torque density. 

In order to confine an $m=1$ spiral between two $Q$-barriers, we
should have $Q^2(1-\nu^2)=1$ at two radii. Assuming 
Keplerian rotation and a slow pattern speed $\Omega_p\ll\Omega$, this amounts to
\begin{align}
  \left(\frac{R_{Qb}}{R_c}\right)^{-3/2} = 2Q^2(R_{Qb}). 
\end{align}
Then it may be possible to have two $Q$-barriers when the $Q$ profile
of the disc rises more rapidly (decays more slowly) than $R^{-3/2}$
for decreasing (increasing) $R$. One possibility is a gap
opened by a giant planet in a constant-$Q$ disc. In that case $Q$
rises rapidly towards the gap centre since it is a region of low
surface density. The outer gap edge is then a potential site for the
growth of a low-frequency one-armed spiral through the background
torque. 

%talk about additional simulations at high Q
%spiral mode as external potential to wider disc (lp11)
%interaction across co-rotation






\subsection{The confined spiral as an external potential}
One possible effect of self-gravity is to allow the one-armed spiral  
to act as an external potential for the wider disc. This is
analogous to disc-satellite interaction 
\citep{goldreich79}. Here, we estimate the magnitude of this effect
using basic results from disc-planet theory \citep[see, e.g.][and
references therein]{papaloizou07}.  

Let us treat the one-armed spiral confined in $R\in[R_1,R_2]$ as an  
external potential of the form $\Phi_\mathrm{ext}(R)\cos{\left(\phi -
    \Omega_pt\right)}$. We take 
\begin{align} 
  \Phi_\mathrm{ext}
  =-\frac{GM_\mathrm{ring}}{\overline{R}}b^{1}_{1/2}(\beta),   
\end{align}
where $M_\mathrm{ring}$ is the disc mass contained within
$R\in[R_1,R_2]$, $\overline{R} = (R_1+R_2)/2$ is the approximate radial
location of the spiral, $b_{n}^m(\beta)$ is the Laplace coefficient
and $\beta = R/\overline{R}$. This form of
$\Phi_\mathrm{ext}$ is the $m=1$ component of the gravitational
potential of an external satellite on a circular orbit
\citep{goldreich79}.    

%For an order-of-magnitude exercise, we simply take 
%\begin{align}
 % \Phi_\mathrm{ext} = -\frac{GM_\mathrm{ring}}{|R - \overline{R}|}, 
%\end{align}
%for the potential associated with a disturbance located at
%$R=\overline{R}$, where $M_\mathrm{ring}$ 

We expect the external potential to exert a torque on the disc at the
Lindblad and co-rotation resonances. At the outer Lindblad resonance
(OLR), this torque is 
\begin{align}
  \Gamma_L =
  \frac{\pi^2\Sigma_L}{3\Omega_L\Omega_p}
  \left[\left.R_L\frac{d\Phi_\mathrm{ext}}{dR}\right|_L + 2\left(1 -
      \frac{\Omega_p}{\Omega_L}\right)\Phi_\mathrm{ext}\right]^2, 
\end{align}   
where a Keplerian disc has been assumed and subscript $L$ denotes
evaluation at the OLR, $R=R_L$. (The inner Lindblad resonance does not
exist for the pattern speeds observed in our simulations.) 

If we associate the external potential with angular momentum 
$J_\mathrm{ext}  = M_\mathrm{ring}\overline{R}^2\Omega_p$, we can
calculate a growth rate $\gamma_L=\Gamma_L/J_\mathrm{ext}$. Then 
\begin{align}
  \frac{\gamma_L}{\Omega_p} = &\frac{\pi
    h}{3Q_L}\left(\frac{M_p}{M_*}\right)\left(\frac{R_L}{\overline{R}}\right)\left(\frac{R_c}{\overline{R}}\right) 
  ^3\left(\frac{R_L}{R_c}\right)^{-3/2}\notag\\ 
  &\times
  \left\{\frac{R_L}{\overline{R}}\left.\frac{db_{1/2}^1}{d\beta}\right|_L
    + 2\left[1 - 
      \left(\frac{R_c}{R_L}\right)^{-3/2}\right]b_{1/2}^1(\beta_L)\right\}^2. 
\end{align}
%\begin{align}
 % \frac{\gamma_L}{\Omega_p} \sim &\frac{\pi h}{3
  %  Q_L}\left(\frac{M_\mathrm{ring}}{M_*}\right)\left(\frac{R_L}{\overline{R}}\right)^{-1/2}\left(\frac{R_c}{\overline{R}}\right)^{5/2}\notag\\
 % &\times \left\{ \frac{R_L}{R_L - \overline{R}} - 2\left[1 -
  %    \left(\frac{R_c}{R_L}\right)^{-3/2}\right]\right\}^2\frac{R_c^2}{\left(R_L
   % - \overline{R}\right)^2}. 
%\end{align}
Inserting $h=0.05$, $Q_L=10$, $M_\mathrm{ring} = 0.05M_*$,
$R_L=7.2R_0$, $R_c=4.4R_0$ and $\overline{R}=1.5R_0$ from our fiducial
FARGO simulation, we get
\begin{align}
  \gamma_L \sim 5\times10^{-4}\Omega_p. 
\end{align}
%\begin{align}
 % \gamma_L \sim 0.012\Omega_p. 
%\end{align}
%This represents an upper limit on the effect of the OLR on the spiral
%disturbance, and is much smaller than the growth rate measured in the
%simulation ($\gamma\sim0.1\Omega_p$). We conclude that for our models,
%the OLR has negligible effect on the one-armed spiral.  
For the co-rotation torque, we use the result
\begin{align}
  \Gamma_c = \left.
    \pi^2\Phi_\mathrm{ext}^2\left(\frac{d\Omega}{dR}\right)^{-1}\frac{d}{dR}\left(\frac{2\Sigma}{\Omega}\right)\right|_{c}     
\end{align}
for a Keplerian disc, where subscript $c$ denotes evaluation at
co-rotation radius $R=R_c$. For a power-law surface density profile
$\Sigma\propto R^{-s}$ we have
\begin{align}
\frac{\gamma_c}{\Omega_p} = \frac{4}{3}\frac{\pi h}{Q_c}
\left(\frac{M_\mathrm{ring}}{M_*}\right)\left(\frac{R_c}{\overline{R}}\right)^4\left(s-\frac{3}{2}\right) 
\left[b_{1/2}^1(\beta_c)\right]^2    
\end{align}
%\begin{align}
%  \frac{\Gamma_C}{J_\mathrm{ext}\Omega_p} = \frac{4}{3}\frac{\pi h}{Q_C} \left(s -
 %   \frac{3}{2}\right)\left(\frac{M_\mathrm{ring}}{M_*}\right)\frac{R_c^4}{\overline{R}^2\left(R_c
 %     - \overline{R}\right)^2}.   
%\end{align}
Using the above parameter values with $s=2$ and $Q_c=10$, we obtain a
rate
\begin{align}
  \gamma_c\sim 6\times 10^{-4}\Omega_p. 
\end{align}

These estimates are much smaller than that due to
the imposed temperature gradient as measured in the simulations
($\gamma\sim 0.1\Omega_p$). We conclude that for our disc models, the
Lindblad and co-rotation resonances have negligible effects on the
growth of the one-armed spiral in the inner disc. We confirmed this with
additional FARGO simulations with a reduced radial domain size,
setting $R_\mathrm{max}=3R_0$, which still developed the one-armed
spiral. 

The torque exerted on the disc at the OLR by an
external potential is positive, while that at co-rotation 
depends on the gradient of potential vorticity there
\citep{goldreich79}. For our disc models with surface density
$\Sigma\propto R^{-2}$ in the outer disc, this co-rotation torque is
positive. Interpreting the one-armed spiral as an external potential
for the outer disc, this means that the spiral loses
angular momentum by launching density waves with positive angular momentum at
the OLR, and by applying a positive co-rotation torque on the disc. 
In principle, this interaction is destabilising because the
one-armed spiral has negative angular momentum
\citep{lin11b}. This effect is small in our simulations, but it  
could be significant in other parameter regimes. %Furthermore, this
%non-local interaction may be important for the long-term evolution of
%the outer disc, since it gains angular momentum from the

%also transport ang mom outwards


{\bf
  \subsection{Relevance to protoplanetary discs}
  
  
}