\section{Discussion}\label{discussions} 
{\bf
  We discuss below several issues related to self-gravitating
  discs in the context of our numerical simulations. However, it is
  important to keep in mind that 
  the growth of the one-armed spiral in our simulations is
  \emph{not} a gravitational instability in the sense that
  destabilisation is through the background torque associated with a
  forced temperature gradient, and not by self-gravitational torques\footnote{In fact, additional simulations with $Q_\mathrm{out}=4$ (giving $Q\gtrsim2.5$ throughout the disc) still 
  develops the one-armed spiral, but with a smaller growth rate.}. 
}

{\bf
  \subsection{Motion of the central star} 
  In our models we have purposefully neglected the indirect potential
  associated with a non-inertial reference frame to avoid
  complications arising from the motion of the central star.  
  Although it has been established that such motion can destabilise an
  $m=1$ disturbance in the disc \citep{adams89,shu90,michael10}, 
  the disc masses in our models ($M_d\lesssim 
  0.1 M_*$) are not expected to be sufficiently massive for this effect to be
  significant. Indeed, simulations including the
  indirect potential, carried out in the early stages of this project,
  produced similar results.    
  
  \subsection{Role of Lindblad and co-rotation torques}
  One effect of self-gravity is to allow the one-armed spiral,
  confined to $R\sim R_0$ in our models, to act as an external potential for the
  exterior disc in $R>R_0$. This is 
  analogous to disc-satellite interaction 
  \citep{goldreich79}, where the embedded satellite exerts a torque on
  the disc at Lindblad and co-rotation resonances. 

  In Appendix \ref{disc-planet} we estimate the magnitude of this effect
  using basic results from disc-planet theory \citep[see, e.g.][and
  references therein]{papaloizou07}. There, we find that the angular
  momentum exchange between the one-armed spiral and the exterior disc
  is insignificant compared to the background torque.  
  
  We confirmed this with additional FARGO simulations that exclude the
  co-rotation and outer Lindblad resonances (OLR) by reducing radial domain
  size to $R_\mathrm{max} = 3 R_0$, which still developed 
  the one-armed spiral. 
  %instability due to local exchange of angular momentum with the
  %background disc 
  %CR and LR do not play a role - extended outer disc not essential
  %(but used here to avoid potential boundary effects) cannot exclude
  %a priori 
}


{\bf
  \subsection{Applicability to protoplanetary discs}
  
  \subsubsection{Thermodynamic requirements}  
  % talk about additional simulations at high Q
  %issues with fragmentation 
 % requirement for fast cooling  
  % fast cooling fragments 
  % stabilised by irradiation
  % background torque and fragmentation may not be mutually exclusive
  % (speculations) 
  % these studies do no account for possible global disk structure 
  %Clarke (2009), Rafikov (2005), and Rice & Armitage (2009)
  %no fragmentation in our case because disc is maintained at slightly
  %higher values of Q. no fragmentation of one arm spiral because
  %co-rotation radius isn't on it 
  %if irradiation 
  %one-armed low frequency spiral probably will not fragment because
  %co-rotation is far away 

  A locally isothermal equation of state represents the ideal limit
  of infinitely short cooling and heating timescales, so the 
  disc temperature instantly returns to its initial value when 
  perturbed. The background torque is generally non-zero if the 
  resulting temperature profile has a non-zero radial gradient.  
  
  A short cooling timescale $t_c$ can occur in the outer
  parts of protoplanetary discs 
  \citep{rafikov05,clarke09,rice09,cossins10b,tsukamoto15}.  %rafikov07 (convection)
  However, if a disc with $Q\simeq 1$ is cooled (towards zero
  temperature) on a timescale $t_c\lesssim\Omega_k^{-1}$, it will
  fragment following gravitational instability
  \citep{gammie01,rice05,paardekooper12}.  
  
  Fragmentation can be avoided if the disc is heated to maintain 
  $Q>Q_c$, the threshold for fragmentation \citep[$Q_c\simeq
  1.4$ for isothermal discs,][]{mayer04}. This may be 
  possible in the outer parts of protoplanetary discs due to
  stellar irradiation \citep{rafikov09,kratter11,zhu12}. Sufficiently strong
  external irradiation is expected to suppress the linear gravitational
  instabilities altogether \citep{rice11}.  

  %chiang and goldreich 
  The background torque may thus exist in the outer
  parts of protoplanetary discs that are irradiated, such
  that the disc temperature is set externally with a non-zero radial
  gradient \citep[e.g.][]{stamatellos08}. Of course, if external irradiation sets a strictly
  isothermal outer disc \citep[e.g.][]{boley09}, then the background
  torque vanishes.      

  % We note, however, that there is no obvious reason why the background
  % torque and gravitational instability/fragmentation cannot coexist.  

  % It is important to note, however,
  % that these studies do not consider the possible effect of global
  % structures present in our numerical simulations.  
  
  % When $Q\simeq 1$ and the cooling timescale
  % $t_c\gtrsim\Omega_k^{-1}$, the disc reaches a
  % quasi-steady in which cooling is balanced by heating due to
  % gravito-turbulence \citep{lodato04}. This would, in general, produce
  % a global radial temperature gradient.   
     
  % where does the structure come from? (mass built up, planet gap, infall) 
 
  \subsubsection{Radial disc structure}
  In our simulations the 
  $m=1$ spiral is confined between two $Q$-barriers, where real solutions to the local
  dispersion relation is possible (Eq. \ref{wavenumber}). The
  existence of such a cavity is a result of the initial surface
  density bump (Eq. \ref{sig_bump}).  
  Thus, in our disc models the main role of disc structure 
  and self-gravity is to allow a local $m=1$ mode to be set up, which is then 
  destabilised by the background torque. 
  
  In order to confine an $m=1$ mode between two $Q$-barriers, we
  should have $Q^2(1-\nu^2)=1$ at two radii. Assuming 
  Keplerian rotation and a slow pattern speed $\Omega_p\ll\Omega$,
  this amounts to 
  \begin{align}\label{qb_cond}
    \left(\frac{R_{Qb}}{R_c}\right)^{-3/2} = 2Q^2(R_{Qb}). 
  \end{align}
  Then two $Q$-barriers may exist when the $Q^2$ profile
  rises more rapidly (decays more slowly) than $R^{-3/2}$
  for decreasing (increasing) $R$. Note that 
  Eq. \ref{qb_cond} does not necessarily require strong self-gravity
  if $R_c$ is large. 
  
  A surface density bump can develop in 
  `dead zones' of protoplanetary 
  discs, where there is reduced mass accretion because the magneto-rotational 
  instability is ineffective for angular momentum transport 
  \citep{gammie96,turner08,landry13}. The dead zone becomes 
  self-gravitating with sufficient mass built-up 
  \citep{armitage01,martin12,martin12b,zhu09,zhu10,zhu10b,bae13}.  
  
  However, conditions in a dead zone may not be suitable for 
  sustaining a background torque because it may not cool/heat fast enough
  to maintain a fixed temperature profile. Recently, 
  \cite{bae14} presented sophisticated numerical models of
  dead zones, including a range of heating and 
  cooling processes, which show that dead zones developed large-scale
  (genuine) gravitational instabilities with multiple spiral
  arms. Although this does not prove absence of the background torque,
  it is probably insignificant compared to gravitational torques. 

  Another possibility is a gap opened by an embedded planet. In that case $Q$
  rises rapidly towards the gap centre since it is a region of low
  surface density. This can satisfy Eq. \ref{qb_cond}. Then the inner
  edge of our bump function mimics the outer gap edge. The outer gap
  edge is then a potential site for the growth of a low-frequency
  one-armed spiral through the background torque. However, the
  locally isothermal requirement would limit this process to the
  outer disc irradiated by the central star, or that the temperature
  profile about the gap edge is set by the planet luminosity.  

  Here, it is worth mentioning transition disc 
  around HD 142527, the outer parts of which displays an $m=1$
  asymmetry \citep{fukagawa13} and spiral arms 
  \citep{christiaens14} just outside a disc gap. These authors estimate 
  $Q\simeq$1---2 in the outer disc, implying self-gravity is
  important, but the disc may remain gravitationally-stable
  \citep{christiaens14}. This is a  
  possible situtation that our disc models represent. 
}


