\section{Discussion}\label{discussions} 
Having confirmed the $m=1$ spiral instability through numerical
simulations, we now use results from the simulations combined with
linear theory to construct an explanation for their growth. For
simplicitly, we return to 2D discs. 


% According to previous studies \citep{adams89}, our disc models
% should not support $m=1$ linear spiral instabilities. This is because
% our discs are not sufficiently massive ($M_d\lesssim 0.1M_*$) and, more
% importantly, we have surpressed the motion of the central star. The 
% latter can induce $m=1$ instabilities in massive discs  \citep{shu90}.   


\subsection{Angular momentum density of  low-frequency $m=1$ modes}
From Eq. \ref{lin_ang_mom_cons} and assuming a time-dependence of the
form $\exp{(-\ii \sigma t)}$ with $\real{\sigma} = \omega$,  
the angular momentum density associated with linear waves is
\begin{align}
  \jlin = \frac{m\Sigma}{2}\left[\left(\omega -
      m\Omega\right)|\bm{\xi}|^2 + \ii\Omega\left(\xi_R\xi_\phi^* -
      \xi_R\xi_\phi^*\right)\right].  
\end{align}
For a low-frequency mode, $|\omega|\ll m\Omega$. Then neglecting the
term $\omega|\bm{\xi}|^2$ and setting $m=1$,
\begin{align}
  \jlin &\simeq \frac{\Sigma\Omega}{2}\left[-|\bm{\xi}|^2 + \ii\left(\xi_R\xi_\phi^* -
      \xi_R\xi_\phi^*\right)\right]\notag\\
  & = -\frac{\Sigma\Omega}{2}|\xi_R + \ii\xi_\phi|^2\notag\\
  & \leq 0.
\end{align}
Thus, low-frequency $m=1$ modes are generally associated with negative
angular momentum. If the mode loses (positive) angular momentum
to the background, then we can expect instability. We demonstrate
this below. 

\subsection{Unstable interaction between low-frequency $m=1$ spirals
  and the background disc due to an imposed temperature gradient}
The $m=1$ spiral that emerges in our simulations have low frequency
and grows slowly, and its spatial form is consistent local theory
(\S\ref{fargo_m1}). To demonstrate possible instability, we
first construct a neutral low-frequency mode using local theory, then
show that the torque density acting on this mode due to the background
temperature gradient is negative, which enforces the mode. % Note that
% we will not be assuming a self-gravitating disc until the end of this
% section. 

We begin by recalling the expression for the torque density due to a
locally isothermal equation of state,
\begin{align}
  T_\mathrm{BG} &= -\frac{m}{2}\frac{dc_s^2}{dR}\imag\left[\delta\Sigma_m \left(\frac{\delta
        v_{Rm}}{-\ii\sbar}\right)^*\right] \notag\\
  & = \frac{m}{2}\frac{dc_s^2}{dR}\imag\left[\ii \delta\Sigma_m \left(\frac{\delta
        v_{Rm}}{\sbar}\right)^*\right]
\end{align}
where we replaced the radial Lagrangian displacement by the Eulerian
radial velocity perturbation in Eq. \ref{baroclinic_torque}. 
The linearised momentum equations give
\begin{align}
  D\delta v_{Rm} =& 
  \ii\sbar\left[c_s^2\frac{d}{dR}\left(\frac{\delta\Sigma_m}{\Sigma}\right)
    + \frac{d}{dR}\delta\Phi_m\right] \notag\\ 
  &- \frac{2\ii
    m\Omega}{R}\left(c_s^2\frac{\delta\Sigma_m}{\Sigma} +
    \delta\Phi_m\right),
\end{align}
where $D\equiv \kappa^2 - \sbar^2$. 

We now invoke local theory by setting $d/dR \to \ii k$ where $k$ is
real, and using the local solution to the Poisson equation
\begin{align}
  \delta\Phi_m = -\frac{2\pi G}{|k|}\delta\Sigma_m 
\end{align}
\citep{shu91}. Then the radial velocity perturbation becomes
\begin{align}
  \delta v_{Rm} =
  \frac{1}{D}\frac{\delta\Sigma_m}{\Sigma}\left(\frac{2\pi G
      \Sigma}{|k|} - c_s^2\right)\left(k\sbar + \frac{2\ii
      m\Omega}{R}\right),
\end{align}
so that 
\begin{align}
  \ii\delta\Sigma_m \left(\frac{\delta v_{Rm}}{\sbar}\right)^* =
  \frac{1}{D^*}\frac{|\delta\Sigma_m|^2}{\Sigma}\left(\frac{2\pi G
      \Sigma}{|k|} - c_s^2\right)\left(\ii k  + \frac{2
      m\Omega}{R\sbar^*}\right). 
\end{align}
For a neutral mode, $\sbar$, and therfore $D$ is real. Then, after
taking the imaginary part of the above expression, the torque
density is 
\begin{align}\label{tbg_explicit}
  T_\mathrm{BG} &=
  \frac{m}{2}\frac{dc_s^2}{dR}\left[\frac{k}{D}\frac{|\delta\Sigma_m|^2}{\Sigma}\left(\frac{2\pi G
      \Sigma}{|k|} - c_s^2\right)\right].
% &\simeq \frac{m}{2}\frac{dc_s^2}{dR}\left[\frac{k}{\left(\kappa^2 -
%         m^2\Omega^2\right)}\frac{|\delta\Sigma_m|^2}{\Sigma}\left(\frac{2\pi
%       G \Sigma}{|k|} - c_s^2\right)\right]. 
\end{align}
%where the second line follows from the low-frequency approximation
%$|\sigma|\ll m\Omega$. 
% To determine the sign of $T_\mathrm{BG}$ for low frequency one-arm spirals,
% we examine each factor separately below for $m=1$:
% \begin{itemize}
% \item We observe trailing spirals in our simulations, so $k>0$. 
% \item The factor $\kappa^2 - \Omega^2>0$ for our self-gravitating disc
%   models in the region where the spiral appears. 
% \item We measure $|k|\sim \pi G\Sigma/c_s^2$, so the last factor in
%   the square brackets is positive. Note that this factor is positive
%   provided $|k| < 2/HQ$ or wavelengths $\lambda > \pi Q H$.  
% \end{itemize}
We now use the local dispersion relation,
Eq. \ref{dispersion}, to make the replacement $D = 2\pi G \Sigma |k| -
k^2c_s^2$ in the above expression. This gives
\begin{align}\label{tbg_simple}
  T_\mathrm{BG} =
  \frac{m}{2}\frac{dc_s^2}{dR}\frac{1}{k}\frac{|\delta\Sigma_m|^2}{\Sigma}.  
\end{align} 
This torque density is negative for trailing waves in discs with
temperature decreasing outwards ($k>0$ and $dc_s^2/dR<0$,
respectively). Note that this conclusion does not rely on the
low-frequency approximation and holds for any $m$.   

However, if the linear disturbance under consideration \emph{is} a
low-frequency $m=1$ 
trailing spiral wave, then it has negative angular
momentum. If $dc_s^2/dR<0$, as is typical for circumstellar or
protoplanetary discs, then $T_\mathrm{BG}<0$, and the background disc
applies a negative torque on the disturbance, which further decreases
its angular momentum. This suggests the spiral amplitude will grow.  


Using $\jlin$ and $T_\mathrm{BG}$, we can obtain a rough estimate of
the growth rate $\gamma$ of linear perturbations as  
\begin{align}
  2\gamma \sim \frac{T_\mathrm{BG}}{\jlin},
\end{align}
where the factor of two accounts for the fact that the angular momentum
density is quadradtic in the linear perturbations. Inserting the above
expressions for $\jlin$ and $T_\mathrm{BG}$ for $m=1$ gives
\begin{align}
  2\gamma \sim
  -\frac{dc_s^2}{dR}\frac{1}{k\Sigma^2\Omega}\frac{|\delta\Sigma_1|^2}{|\xi_R
    + \ii\xi_\phi|^2}.
\end{align}
Now, using $\delta\Sigma_m/\Sigma = -\ii k \xi_R - \ii m \xi_\phi/R$
in the local approximation, we have
\begin{align}
  2\gamma \sim -\frac{dc_s^2}{dR}\frac{k}{\Omega}\frac{|\xi_R +
    \xi_\phi/kR|^2}{|\xi_R + \ii \xi_\phi|^2}
  \lesssim -\frac{dc_s^2}{dR}\frac{k}{\Omega}.
\end{align}
where we used $dc_s^2/dR <0$, $k>0$ and $|kR|\gg 1$ to obtain the
inequality. For $c_s^2 = c_{s0}^2 (R/R_0)^{-q}$ adopted in our disc
models,
\begin{align}\label{theoretical_rate}
  2\gamma \lesssim q\frac{c_s^2}{R}\frac{k}{\Omega}. 
\end{align}
Now we consider perturbations in a self-gravitating disc with characteristic
wavenumber $k \sim \pi G \Sigma/c_s^2 \sim \Omega/Q c_s$ (as seen in
our simulations). Then
\begin{align}
  \gamma \lesssim \frac{qh}{2Q}\Omega,
\end{align}
where we used $c_s/R\sim h\Omega$. For our fiducial disc models with
$q=1$, $Q=O(1)$, $h=0.05$, this gives $\gamma/\Omega = O(10^{-2})$,
consistent with simulation results. 

Of course, we have been assuming a neutral mode to obtain expressions
for $\jlin$ and $T_\mathrm{BG}$ in the first place, so
Eq. \ref{theoretical_rate} is only an order-of-magnitude
constraint. Nevertheless, the xpected growth rate is in reasonable
agreement with numerical results. 

%also total torque is integral -> may get cancelations 

\subsection{The role of self-gravity and disc structure} 
The background torque density $T_\mathrm{BG}$ does not depend on  
self-gravity explicitly. In the above discussion, we only considered a self-gravitating perturbation to 
obtain an appropriate wavenumber $k$ needed to evaluate
Eq. \ref{theoretical_rate}. Self-gravity is not directly
responsible for instability. However, 
our numerical simulations show that the $m=1$ spiral is
confined between two $Q$-barriers, where real solutions to the local
dispersion relation is possible (Eq. \ref{wavenumber}), as a result of the adopted disc
structure. Thus, in our disc models the main role of disc structure and self-gravity is to 
allow a neutral mode to be set up and confined, which is then
destabilised through the background torque density. 

In order to confine an $m=1$ spiral between two $Q$-barriers, we
should have $Q^2(1-\nu^2)=1$ at two radii. Assuming 
Keperlian rotation and a slow pattern speed $\Omega_p\ll\Omega$, this amounts to
\begin{align}
  \left(\frac{R_{Qb}}{R_c}\right)^{-3/2} = 2Q^2(R_{Qb}). 
\end{align}
Then it may be possible to have two $Q$-barriers when the $Q$ profile
of the disc rises more rapidly (decays more slowly) than $R^{-3/2}$
for decreasing (increasing) $R$. One possibility is a gap
opened by a giant planet in a constant-$Q$ disc. In that case $Q$
rises rapidly towards the gap centre since it is a region of low
surface density.  %multi spirals associated with one planet 




%talk about additional simulations at high Q
%spiral mode as external potential to wider disc (lp11)
%interaction across co-rotation


\subsection{The confined spiral as an external potential}
One possible effect of self-gravity is to allow the one-arm spiral  
to act as an external potential for the wider disc. This is
analogous to disc-satellite interaction 
\citep{goldreich79}, and may lead to instability 
if the angular momentum associated with the spiral has the opposite 
sign to that of the density waves it induces in the exterior disc
\citep{lin11b}. Here, we estimate the magnitude of this effect using
basic results from disc-planet theory \citep[see, e.g.][and references 
therein]{papaloizou07}. 

Let us treat the one-arm spiral confined in $R\in[R_1,R_2]$ as an  
external potential of the form $\Phi_\mathrm{ext}(R)\cos{\left(\phi -
    \Omega_pt\right)}$. We take 
\begin{align} 
  \Phi_\mathrm{ext}
  =-\frac{GM_\mathrm{ring}}{\overline{R}}b^{1}_{1/2}(\beta),   
\end{align}
where $M_\mathrm{ring}$ is the disc mass contained within
$R\in[R_1,R_2]$, $\overline{R} = (R_1+R_2)/2$ is the approximate radial
location of the spiral, $b_{n}^m(\beta)$ is the Laplace coefficient
and $\beta = R/\overline{R}$. This form of
$\Phi_\mathrm{ext}$ is the $m=1$ component of the gravitational
potential of an external satellite on a circular orbit
\citep{goldreich79}.    

%and here we are interested in the
%potential it produces at $R>\overline{R}$  
%This potential is associated with a mass
%$M_\mathrm{ring}$ contained in   is the mass in $R\in[R_1,R_2]$.

%For an order-of-magnitude exercise, we simply take 
%\begin{align}
 % \Phi_\mathrm{ext} = -\frac{GM_\mathrm{ring}}{|R - \overline{R}|}, 
%\end{align}
%for the potential associated with a disturbance located at
%$R=\overline{R}$, where $M_\mathrm{ring}$ 

We expect the external potential to exert a torque on the disc at the
Lindblad and co-rotation resonances. At the outer Lindblad resonance
(OLR), this torque is 
\begin{align}
  \Gamma_L =
  \frac{\pi^2\Sigma_L}{3\Omega_L\Omega_p}
  \left[\left.R_L\frac{d\Phi_\mathrm{ext}}{dR}\right|_L + 2\left(1 - \frac{\Omega_p}{\Omega_L}\right)\Phi_\mathrm{ext}\right]^2,
\end{align}   
where a Keperian disc has been assumed and subscript $L$ denotes
evaluation at the OLR, $R=R_L$. (The inner Lindblad resonance does not
exist for the pattern speeds observed in our simulations.) 

If we associate the external potential with angular momentum 
$J_\mathrm{ext}  = M_\mathrm{ring}\overline{R}^2\Omega_p$, we can obtain a
rate $\gamma_L=\Gamma_L/J_\mathrm{ext}$. Then 
\begin{align}
  \frac{\gamma_L}{\Omega_p} = &\frac{\pi
    h}{3Q_L}\left(\frac{M_p}{M_*}\right)\left(\frac{R_L}{\overline{R}}\right)\left(\frac{R_c}{\overline{R}}\right)^3\left(\frac{R_L}{R_c}\right)^{-3/2}\notag\\
  &\times \left\{\frac{R_L}{\overline{R}}\left.\frac{db_{1/2}^1}{d\beta}\right|_L + 2\left[1 -
      \left(\frac{R_c}{R_L}\right)^{-3/2}\right]b_{1/2}^1(\beta_L)\right\}^2.
\end{align}
%\begin{align}
 % \frac{\gamma_L}{\Omega_p} \sim &\frac{\pi h}{3
  %  Q_L}\left(\frac{M_\mathrm{ring}}{M_*}\right)\left(\frac{R_L}{\overline{R}}\right)^{-1/2}\left(\frac{R_c}{\overline{R}}\right)^{5/2}\notag\\
 % &\times \left\{ \frac{R_L}{R_L - \overline{R}} - 2\left[1 -
  %    \left(\frac{R_c}{R_L}\right)^{-3/2}\right]\right\}^2\frac{R_c^2}{\left(R_L
   % - \overline{R}\right)^2}. 
%\end{align}
Inserting $h=0.05$, $Q_L=10$, $M_\mathrm{ring} = 0.05M_*$,
$R_L=7.2R_0$, $R_c=4.4R_0$ and $\overline{R}=1.5R_0$ from our fiducial
FARGO simulation, we get
\begin{align}
  \gamma_L \sim 5\times10^{-4}\Omega_p. 
\end{align}
%\begin{align}
 % \gamma_L \sim 0.012\Omega_p. 
%\end{align}
%This represents an upper limit on the effect of the OLR on the spiral
%disturbance, and is much smaller than the growth rate measured in the
%simulation ($\gamma\sim0.1\Omega_p$). We conclude that for our models,
%the OLR has negligible effect on the one-arm spiral.  
For the co-rotation torque, we use the result
\begin{align}
  \Gamma_c = \left.
    \pi^2\Phi_\mathrm{ext}^2\left(\frac{d\Omega}{dR}\right)^{-1}\frac{d}{dR}\left(\frac{2\Sigma}{\Omega}\right)\right|_{c}   
\end{align}
for a Keplerian disc, where subscript $c$ denotes evaluation at
co-rotation radius $R=R_c$. For a power-law surface density profile
$\Sigma\propto R^{-s}$ we have
\begin{align}
\frac{\gamma_c}{\Omega_p} = \frac{4}{3}\frac{\pi h}{Q_c}
\left(\frac{M_\mathrm{ring}}{M_*}\right)\left(\frac{R_c}{\overline{R}}\right)^4\left(s-\frac{3}{2}\right)
\left[b_{1/2}^1(\beta_c)\right]^2   
\end{align}
%\begin{align}
%  \frac{\Gamma_C}{J_\mathrm{ext}\Omega_p} = \frac{4}{3}\frac{\pi h}{Q_C} \left(s -
 %   \frac{3}{2}\right)\left(\frac{M_\mathrm{ring}}{M_*}\right)\frac{R_c^4}{\overline{R}^2\left(R_c
 %     - \overline{R}\right)^2}.   
%\end{align}
Using the above parameter values with $s=2$ and $Q_c=10$, we obtain a
rate
\begin{align}
  \gamma_c\sim 6\times 10^{-4}\Omega_p. 
\end{align}

These estimates are much smaller than that due to
the imposed temperature gradient as measured in the simulations
$\gamma\sim 0.1\Omega_p$). We conclude that for our disc models, the
Lindblad and co-rotation resonances have negligible effect on the
growth of the spiral arm in the inner disc. We confirmed this with
additional FARGO simulations with a reduced radial domain size,
setting $R_\mathrm{max}=3R_0$, which still developed the one-arm
spiral. 

% both are positive -> destabilizing in principle 

%\begin{align}
%  \gamma_C\sim 0.010\Omega_p,
%\end{align}
%which is again small compared to the measured growth rates. 

%not important for our models, but may be important if outer disc is
%self-gravitating too

%expressions for linear response, but have massive "planet" -> but CR
%and LR far from where spiral is 

%required disc structure (for confined SG spiral)

%\subsection{Speculations}
%speculations - long term balance. forced temp gradient drives
%instability (disc gains ang mom, spiral loses ang mom). but if spiral
%shocks and dissiple, it dumps negative angular momentum (disc gains
%negative ang mom). balance between gain from forced temp gradient and
%loss due to dissiplatoin

%other source of m=1 disturbance -> eccentric modes (adams, PJ06,
%hopkins, tremaine)

