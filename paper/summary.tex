\section{Summary and conclusions}\label{summary}
In this paper, we have carried out direct numerical hydrodynamic
simulations of radially structured, self-gravitating locally
isothermal discs.     
We find such systems can be unstable to low-frequency perturbations
with azimuthal wavenumber $m=1$, which leads to the development of an one-armed 
trailing spiral that persist for at least $O(10^2)$ orbits. The spiral
pattern speed is smaller than the local disc rotation and
growth rates are $O(10^{-2}\Omega)$ which gives a characteristic
growth time of $O(10)$ orbits.   
%While this is slow
%compared to the local rotation, it is dynamical at co-rotation.  

We used three independent numerical codes --- FARGO in 2D, ZEUS-MP and
PLUTO in 3D --- to show that the growth of  
one-armed spirals in our disc models is due to the imposed temperature 
profile and disc structure. Growth rates increased 
with the magnitude of the imposed temperature gradient, and one-armed
spirals did not develop in strictly isothermal simulations. The spiral
is confined between two $Q$-barriers owing to a surface density bump. 
We find the instability behaves similarly in 2D and 3D. However, in 3D
the spiral disturbance can become radially global away from 
the midplane.  

We applied angular momentum conservation within linear theory to
explain our numerical results. We find a   
forced temperature gradient introduces a torque on linear 
perturbations. We called this the background torque because it
represents an exchange of angular momentum between the background disc
and the perturbations. This offers a previously unexplored pathway to
instability. 

In the local approximation, we showed that this background torque is
negative for trailing spirals in discs with a fixed temperature or
sound-speed profile that decrease outwards. This torque enforces  
low-frequency $m=1$ modes because they are associated with negative
angular momentum. We therefore interpret the one-armed spiral instability
observed in our simulations as an initially neutral, tightly-wound
$m=1$ mode being destabilised by the imposed temperature gradient. 
   
\subsection{Speculations and future work}
%numerical simulations to include cooling/heating
A locally isothermal equation of state represents the limit of
infinitely short cooling (and heating) timescales, so the disc
temperature instantly returns to its initial value when perturbed.  
This assumption can be relaxed by including an energy equation with
thermal relaxation. Preliminary FARGO simulations
indicate a thermal relaxation timescale $t_c < 0.1\Omega_k^{-1}$ is
needed for the one-armed spiral to develop. However, this value may
depend on disc parameters, and will be explored in a follow-up  
study.  

% prelim sims show they are long lasting 
% balancing heating and cooling 
% should not get fragmentation (CR away from spiral) 
In the deeply non-linear regime, the one-armed spiral may 
shock and deposit negative angular momentum onto
the background disc. The spiral amplitude would saturate by gaining
positive angular momentum. However, if the temperature gradient is
maintained, it may be possible to achieve a balance between the gain
of negative angular momentum through the background torque, and the
gain of positive angular momentum through shock dissipation. We remark  
that fragmentation is unlikely because the co-rotation radius 
is outside the bulk of the spiral arm, in the non-self-gravitating
portion of the disc \citep{durisen08}. In order
to study these possibilities, improved numerical models are needed to
ensure total angular momentum conservation on timescales much longer
than that considered in this paper. 

% non-sg sims, Be stars, eccentric discs
% We stress that destabilisation through the temperature gradient
% does not explicitly depend on self-gravity. In principle, this effect
% will destabilise low-frequency $m=1$ trailing spirals regardless of
% its origin.

In our models and interpretation, self-gravity coupled with the disc 
structure permits a neutral one-armed mode, which is then
destabilised by the temperature gradient. Other origins of the neutral
one-armed spiral should be investigated. One possibility is in Be star discs
\citep{rivinius13}, for which 
one-armed oscillations may explain long-timescale variations in their
emission lines \citep[see e.g.][and references
therein]{okasaki97,papaloizou06c,ogilvie08}. These studies employ
strictly isothermal disc models, so it would be interesting to explore
the effect of a temperature gradient on the stability of these
oscillations. 

%vertical shear instability?

%linear theory: explicit calculations 
%For a self-consistent description of the instability, one needs to
%explicitly solve for linear stability problem. 

%theoretical investigation on effect on other m modes