\section{Summary and conclusions}\label{summary}
In this paper, we have carried out direct numerical hydrodynamic
simulations of radially structured, self-gravitating locally
isothermal discs.     
We find such systems can be unstable to low-frequency perturbations
with azimuthal wavenumber $m=1$, which lead to the development of one-armed 
trailing spirals that persist for at least $O(10^2)$ orbits. The 
spiral pattern speed is smaller than the local disc rotation, making
it almost stationary. The growth rates are $O(10^{-2}\Omega)$ which
gives a characteristic growth time of $O(10P_0)$. 
%While this is slow
%compared to the local rotation, it is dynamical at co-rotation.  

We used three independent numerical codes --- FARGO in 2D, ZEUS-MP and
PLUTO in 3D --- to verify the growth of  
one-armed spirals in our disc models is due to the imposed temperature
profile and disc structure. The spiral is confined between two
$Q$-barriers owing to a surface density bump. Growth rates increased
with the magnitude of the imposed temperature gradient, and one-armeded
spirals did not develop in strictly isothermal simulations.  
We find no significant vertical 
motion in 3D simulations, which indicates the instability is
two-dimensional. 

We applied angular momentum conservation within linear theory to
explain our numerical results. A  
forced temperature gradient introduces a torque on linear 
perturbations. We called this the background torque because it
represents an exchange of angular momentum between the background disc
and the perturbations. In the local approximation, we showed that this
background torque is negative for trailing spirals, which enforces  
low-frequency $m=1$ modes because they are associated with negative
angular momentum. We therefore interpret the one-armed spiral instability
observed in our simulations as a neutral, tightly-wound $m=1$ mode
being destabilised by the imposed temperature gradient.  

%works in 2D/3D
%forced gradient feedback 
%neutral mode destabilized      
\subsection{Spectulations and future work}
%accretion??
%balancing heating and cooling 
In the deeply non-linear regime, the one-armed spiral may 
shock and dissipate, depositing negative angular momentum onto
the background disc. The spiral amplitude would saturate by gaining
positive angular momentum. However, if the temperature gradient is
forced, it may be possible to achieve a balance between the gain of negative
angular momentum through the forced temperature gradient, and the gain
of positive angular momentum through shock dissipation. In order to
study this possibility, improved numerical models are needed to ensure 
total angular momentum conservation in long-term simulations.  

%numerical simulations to include cooling/heating
A locally isothermal disc represents the limit of infinitely short
cooling (and heating) timescales, so the disc temperature instantly
returns to its inital value. %so we don't go to deeply non-linear
                             %regime 
This assumption can be relaxed by including an energy
equation with thermal relaxation. Preliminary FARGO simulations
indicate a relaxation timescale $t_c < 0.1\Omega_k^{-1}$ is needed for
the one-armed spiral to develop. This will be presented in a follow-up
study. %prelim sims show they are long lasting 

%non-sg sims, Be stars, eccentric discs
% We stress that destabilisation through the temperature gradient
% does not explicitly depend on self-gravity. In principle, this effect
% will destabilise low-frequency $m=1$ trailing spirals regardless of
% its origin.
In our models and interpretation, self-gravity coupled with the disc 
structure permits a neutral one-armed mode, which is then
destabilised. Other origins of a neutral
one-armed spiral should be investigated. One possibility is in Be star discs
\citep{rivinius13}, for which 
one-armed oscillations may explain long-timescale variations in their
emission lines \citep[see e.g.][and references
therein]{okasaki97,papaloizou06c,ogilvie08}. These studies employ
strictly isothermal disc models, so it would be interesting to explore
the effect of a temperature gradient on the stability of these
oscillations. 


%vertical shear instability?

%linear theory: explicit calculations 
%For a self-consistent description of the instability, one needs to
%explicitly solve for linear stability problem. 