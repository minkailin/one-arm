\section{Summary and conclusions}\label{summary}
In this paper, we carried out direct hydrodynamic simulations of 
radially structured, self-gravitating locally isothermal discs. 
We find such systems can be unstable to low-frequency perturbations
with azimuthal wavenumber $m=1$, which lead to the development of one-arm 
trailing spirals that persist for at least $O(10^2)$ orbits. The 
spiral pattern speed is smaller than the local disc rotation, making
it almost stationary. 

According to previous studies \citep{adams89}, our disc models
should not support $m=1$ linear spiral instabilities. This is because
our discs are not sufficiently massive ($M_d\lesssim 0.1M_*$) and, more
importantly, we have surpressed the motion of the central star. The 
latter can induce $m=1$ instabilities in massive discs  \citep{shu90}.    

In order to explain our numerical results, we invoked   
angular momentum conservation within linear theory. A    
forced temperature gradient introduces a torque on linear
perturbations.  We showed that when the linear perturbation is a
low-frequency $m=1$ mode, 



%works in 2D/3D
%forced gradient feedback 
%neutral mode destabilized      

\subsection{Spectulations and future work}


%numerical simulations to include cooling/heating
%non-sg sims, Be stars, eccentric discs
%linear theory: explicit calculations 