 % Strictly speaking, the growth of the one-armed spiral
 %   observed in our simulations is \emph{not} a gravitational instability
 %   because destabilisation is through the forced temperature gradient
 %   and not disc self-gravity.  Nevertheless, we discuss below several issues 
 %   related to self-gravitating discs.

\section{Summary and conclusions}\label{summary}
{\bf
  In this paper, we have described a destabilising  
  effect of adopting a fixed temperature profile to model   
  astrophysical discs. By applying angular momentum conservation
  within linear theory, we showed that a forced temperature gradient 
  introduces a torque on linear perturbations. We call this the
  background torque because it represents an exchange of angular
  momentum between the background disc and the perturbations. This 
  offers a previously unexplored pathway to instability in locally
  isothermal discs.  
}

In the local approximation, we showed that this background torque is
negative for {\bf non-axisymmetric} trailing waves in discs with a fixed temperature or
sound-speed profile that decrease outwards. {\bf The background torque} enforces  
low-frequency {\bf non-axisymmetric} modes because they are associated
with negative angular momentum.

{\bf 
  We demonstrated the destabilising effect of the background torque by
  carrying out direct numerical hydrodynamic simulations of 
  locally isothermal discs with a self-gravitating surface density
  bump. 
}
We find such systems {\bf are} unstable to low-frequency perturbations
with azimuthal wavenumber $m=1$, which leads to the development of an one-armed 
trailing spiral that persist for at least $O(10^2)$ orbits. The spiral
pattern speed is smaller than the local disc rotation and
growth rates are $O(10^{-2}\Omega)$ which gives a characteristic
growth time of $O(10)$ orbits. 

We used three independent numerical codes --- FARGO in 2D, ZEUS-MP and
PLUTO in 3D --- to show that the growth of     
one-armed spirals in our disc model is due to the imposed
temperature {\bf gradient: growth rates increased linearly}  
with the magnitude of the imposed temperature gradient, and one-armed
spirals did not develop in strictly isothermal simulations. {\bf This 
  one-armed spiral instability can be interpreted as an initially 
  neutral, tightly-wound $m=1$ mode being destabilised by the 
  background torque.} The spiral
is mostly confined between two $Q$-barriers surface density bump. 
We find the instability behaves similarly in 2D and 3D {\bf, but in 3D}
the spiral disturbance becomes more radially global away from    
the midplane. 

\subsection{Speculations and future work}
% numerical simulations to include cooling/heating
% A locally isothermal equation of state represents the limit of
% infinitely short cooling (and heating) timescales, so the disc
% temperature instantly returns to its initial value when perturbed.

{\bf There are several issues that remain to be addressed in future
  works:} 

{\bf \emph{Thermal relaxation.} The locally isothermal assumption 
} can be relaxed by including an energy equation with
{\bf a source term that restores the disc temperature over a
  characteristic timescale $t_\mathrm{relax}$}. Preliminary FARGO
simulations indicate a thermal relaxation timescale $t_\mathrm{relax} <
0.1\Omega_k^{-1}$ is needed for the one-armed spiral to
develop. However, this value may {\bf be model-dependent} and will be
explored in a follow-up study. {\bf For example,
a longer $t_\mathrm{relax}$ may be permitted with larger temperature
gradients.} 


% prelim sims show they are long lasting 
% balancing heating and cooling 
% should not get fragmentation (CR away from spiral) 

{\bf \emph{Non-linear evolution}.} In the deeply non-linear regime, the
one-armed spiral may  
shock and deposit negative angular momentum onto 
the background disc. The spiral amplitude would saturate by gaining
positive angular momentum. However, if the temperature gradient is
maintained, it may be possible to achieve a balance between the gain
of negative angular momentum through the background torque, and the
gain of positive angular momentum through shock dissipation. We remark  
that fragmentation {\bf of a low-frequency spiral} is unlikely
because the co-rotation radius is outside the bulk of the spiral arm, in the non-self-gravitating
portion of the disc \citep{durisen08}. In order
to study these possibilities, improved numerical models are needed to
ensure total angular momentum conservation on timescales much longer
than that considered in this paper. 




% non-sg sims, Be stars, eccentric discs
% We stress that destabilisation through the temperature gradient
% does not explicitly depend on self-gravity. In principle, this effect
% will destabilise low-frequency $m=1$ trailing spirals regardless of
% its origin.

{\bf The background torque in other contexts.}  
{\bf In our simulations}, self-gravity coupled with the disc 
structure permits a neutral one-armed mode, which is then
destabilised by the {\bf fixed} temperature gradient.  
Other origins of the neutral one-armed spiral should be
investigated. One possibility is in Be star discs \citep{rivinius13},
for which one-armed oscillations may explain long-timescale variations
in their emission lines \citep[see e.g.][and references 
therein]{okasaki97,papaloizou06c,ogilvie08}. These studies employ
strictly isothermal disc models, so it would be interesting to explore
the effect of a temperature gradient on the stability of these
oscillations. 

%vertical shear instability?
%linear theory: explicit calculations 
%For a self-consistent description of the instability, one needs to
%explicitly solve for linear stability problem. 
%theoretical investigation on effect on other m modes